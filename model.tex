%% The following is a directive for TeXShop to indicate the main file
%%!TEX root = diss.tex

\chapter{Modelling the Sigma Delta Modulator}
\label{ch:Modelling}

In order to apply an optimization framework to the design of the \gls{LF}, the system from Figures \ref{fig:basic-struct-dt} and \ref{fig:basic-struct-ct} must be placed in a form that allows tractable application of the desired performance and stability targets. This includes omission of blocks that have minimal or no effect on the loop as well as linearization of the quantizer. The \gls{AAF} (when present) can be considered as a pre-filter operating on the input signal. The filter $H_0(\lambda)$ that serves as an additional degree of freedom for the \gls{STF} can be set to unity for the purposes of the model. These two filters are not required in stability analysis, because the \gls{NTF} depends only on $H_1(\lambda)$ as seen in \autoref{eq:t}. After noise rejection performance has been optimized, $H_0(\lambda)$ can be tuned as necessary to ensure that the combined gain of the \gls{AAF} and \gls{LF} is close to unity in the signal band. In a similar way, the \gls{DSP} in the output path serves only to filter out the signal and decimate to the original sampling frequency which may be dealt with separately without impacting loop stability.

\section{Linearization of the Quantizer Element}
\label{sec:model-linq}

The nonlinear nature of the quantizer must be captured by the model as a way to enforce stability. As mentioned before, a common linearization approach is to replace the quantizer with an additive noise source \gls{d}. Furthermore, the linear model can incorporate a variable gain \gls{K}. The inclusion of \gls{K} has uses in linearization, stability, and performance that will be expanded upon in \autoref{ch:Stability}. After these simplifications, the block diagram in \autoref{fig:sdm-model} is obtained, which is applicable to \gls{DT} or \gls{CT} designs. In the \gls{DT} case, the loop is operating entirely in the oversampled domain and the \gls{S/H} block is not shown. In the \gls{CT} case, the \gls{S/H} block in the loop is neglected so that \gls{S} and \gls{T} are \gls{CT} transfer functions\footnote{Note that regarding \autoref{fig:sdm-model} and Figures~\ref{fig:basic-struct-dt}/\ref{fig:basic-struct-ct}, the \gls{NTF} $S(\lambda)$ is the same transfer function whether interpreted from $d \rightarrow y$ or $r \rightarrow e$.}.

\begin{figure}
	\centering
	% General Sigma Delta Modulator
	\begin{tikzpicture}[ampersand replacement=\&,scale=0.75, every node/.style={scale=0.75}]
	
		% Place nodes using a matrix
		\matrix (m1) [row sep=2.5mm, column sep=5mm]
		{
			%-----------------------------------------------------------------------------------------------------------------------------------------------------------------------
			\node[coordinate]											(m01) {};							\&
			\node[coordinate]											(m02) {};							\&
			\node[coordinate]											(m03) {};							\&
			\node[coordinate]											(m04) {};							\&
			\node[dspnodeopen,dsp/label=above]								(m05) {$d$};							\& \\
			%-----------------------------------------------------------------------------------------------------------------------------------------------------------------------
			\node[dspnodeopen,dsp/label=left]								(m11) {$r$};							\&
			\node[dspadder,label={below left:$-$}]							(m12) {};							\&
			\node[dspfilter]											(m13) {$H_1(\lambda)$};					\&
			\node[dspgain,fill=white,label={[align=center,yshift=-5]below:Linearized\\Gain}]	(m14) {$K$};							\&
			\node[dspsquare,label={below:Quantizer}]							(m15) {\RaisingEdge};					\&
			\node[dspnodefull]											(m16) {};							\&
			\node[dspnodeopen,dsp/label=right]								(m17) {$y$};							\& \\
			%-----------------------------------------------------------------------------------------------------------------------------------------------------------------------
			\node[coordinate]											(m21) {};							\&
			\node[coordinate]											(m22) {};							\& 
			\node[coordinate]											(m23) {};							\& 
			\node[coordinate]											(m24) {};							\& 
			\node[coordinate]											(m25) {};							\& 
			\node[coordinate]											(m26) {};							\& 
			\node[coordinate]											(m27) {};							\& \\
			%-----------------------------------------------------------------------------------------------------------------------------------------------------------------------
		};
	
		\node[draw,inner xsep=15pt,inner ysep=10pt,dashed,fit={($(m05.north)+(-0.5, 0.7)$) ($(m15.south)+(0.5, -0.6)$)},label={[align=center]above:Linear Model}] {};
		%\node[draw,inner xsep=15pt,inner ysep=10pt,dashed,fit={($(m02.north west)+(-0.25, 0.25)$) ($(m13.south east)+(0.4, -0.5)$)},label=below:{Loop Filter $H(\lambda)$}] {};
	
		\begin{pgfonlayer}{bg}
			\draw[->]		($(m14) + (-0.5,-0.5)$) -- ($(m14) + (0.5,0.5)$);
		\end{pgfonlayer}
		\draw[dspconn] 	(m11) -- (m12);
		\draw[dspconn] 	(m12) -- (m13);
		\draw[dspconn] 	(m13) -- node[midway,above] {$u$} (m14);
		\draw[dspconn]	(m14) -- (m15);
		\draw[dspconn] 	($(m14.east)+(6pt, 0)$) -- (m15);
		\draw[dspline]	(m15) -- (m16);
		\draw[dspconn] 	(m16) -- (m17);
		\draw[dspline] 	(m16) -- (m26);
		\draw[dspline] 	(m26) -- (m22);
		\draw[dspconn]	(m22) -- (m12);
		\draw[dspconn] 	(m12) -- node[midway,above] {$e$} (m13);
		\draw[dspconn] 	(m05) -- (m15);
		\draw[Gray, ->, out=40, in=90, looseness=0.85] ($(m11)+(0, 0.4)$) to node[below, xshift=-22pt, yshift=-5pt] {$T(\lambda)$} ($(m17)+(0.35, 0.4)$);
		\draw[Gray, ->, out=-45, in=135, looseness=1] ($(m05)+(0.35, 0.25)$) to node[above, xshift=5pt] {$S(\lambda)$} ($(m17)+(0.25, 0.35)$);
	
	\end{tikzpicture}
	\caption{The linearized sigma delta loop block diagram with omission of extraneous filters and the quantizer replaced by a variable gain and additive quantization noise signal.}  \label{fig:sdm-model}
\end{figure}

\section{Well-Posedness and Internal Stability}
\label{sec:model-wp-is}

The meaningful application of feedback to reduce an uncertainty (in this case, error introduced by the nonlinear quantizer) requires that the system be well-posed in order for a solution to exist. \autoref{fig:sdm-model} can undergo block diagram manipulation bringing it into the standard feedback form shown in \autoref{fig:sdm-stdf} with signals \gls{r}, \gls{e}, \gls{d}, and \gls{y}.

\begin{figure}[h]
	\centering
	% General Sigma Delta Modulator
	\begin{tikzpicture}[ampersand replacement=\&,scale=0.75, every node/.style={scale=0.75}]
	
		% Place nodes using a matrix
		\matrix (m1) [row sep=2.5mm, column sep=5mm]
		{
			%--------------------------------------------------------------------------------
			\node[coordinate]					(m00) {};		\&
			\node[coordinate]					(m01) {};		\&
			\node[dspfilter]					(m02) {$-1$};	\&
			\node[dspadder]					(m03) {};		\&
			\node[dspnodeopen,dsp/label=right]		(m04) {$r$};		\& \\
			%--------------------------------------------------------------------------------
			\node[dspnodeopen,dsp/label=left]		(m10) {$d$};		\&
			\node[dspadder]					(m11) {};		\&
			\node[dspfilter]					(m12) {$H_1K$};	\&
			\node[coordinate]					(m13) {};		\& \\
			%--------------------------------------------------------------------------------
		};
	
		\draw[dspconn]	(m01) -- (m02);
		\draw[dspconn] 	(m02) -- (m03);
		\draw[dspconn] 	(m04) -- (m03);
		\draw[dspline] 	(m01) to node[midway,left] {$y$} (m11);
		\draw[dspline]	(m03) to node[midway,right] {$e$} (m13);
		\draw[dspconn] 	(m10) -- (m11);
		\draw[dspconn]	(m12) -- (m11);
		\draw[dspconn]	(m13) -- (m12);
	
	\end{tikzpicture}
	\caption{The linearized model converted into standard feedback form.}  \label{fig:sdm-stdf}
\end{figure}

With some abuse of notation, the equations describing this loop are:

\begin{equation}
	\begin{bmatrix}
		r \\
		d
	\end{bmatrix} =
	\begin{bmatrix}
		1 & 1 \\
		-H_1K & 1
	\end{bmatrix}
	\begin{bmatrix}
		e \\
		y
	\end{bmatrix}. \label{eq:stdf}
\end{equation}

A feedback system is considered well-posed if the inverse of the transfer matrix in \autoref{eq:stdf} exists and each of its elements are proper. \autoref{eq:stdf-inv} shows that this is the case if both $S(\lambda)$ and $T(\lambda)$ are proper transfer functions.

\begin{align}
	\begin{bmatrix}
		e \\
		y
	\end{bmatrix} =
	\begin{bmatrix}
		1 & 1 \\
		-H_1K & 1
	\end{bmatrix}^{-1} &=
	\begin{bmatrix}
		\frac{1}{1 + H_1K} & \frac{-1}{1 + H_1K} \\
		\frac{H_1K}{1 + H_1K} & \frac{1}{1 + H_1K}
	\end{bmatrix}
	\begin{bmatrix}
		r \\
		d
	\end{bmatrix} \\ &= 
	\begin{bmatrix}
		S & -S \\
		T & S
	\end{bmatrix}
	\begin{bmatrix}
		r \\
		d
	\end{bmatrix}. \label{eq:stdf-inv}
\end{align}

The principle of internal stability is stricter than \gls{BIBO} stability because it guarantees that the internal states of the system remain bounded. The system in \autoref{eq:stdf-inv} is internally stable if each element of the transfer matrix belongs to the set $\mathcal{R}\mathcal{H}_\infty$, i.e., the set of stable real rational proper transfer functions.

\subsection{Constraints on the \titlecap{\glsentrylong{NTF}}}
\label{sec:model-ntf-constraints}

A sufficient condition for $S(\lambda)$ and $T(\lambda)$ to be proper is that transfer function $H_1(\lambda)$ is a strictly proper real rational transfer function. Internal stability of the system follows if $S(\lambda)$ and $T(\lambda)$ are stable. This leads to the following constraints on the \gls{NTF}:

\begin{enumerate}
	\item $S(\lambda)$ is stable, and \label{it:con-1},
	\item The following equivalent conditions hold: \label{it:con-2}
		\begin{enumerate}
			\item $S(\infty) = 1$,
			\item If $S(\lambda)$ is placed in state-space form, the feedthrough matrix $D=1$, and,
			\item The first element of the impulse response of $S(\lambda)$ is one.
		\end{enumerate}
\end{enumerate}

Most prior work in the area performs optimization directly on the \gls{NTF} of the system. This is effective because it is a relatively accurate model of the noise shaping performance. In addition, constraint \ref{it:con-2} enforces causality on the feedback loop ensuring the system is physically realizable.

\section{Modelling Uncertain Quantizer Gain}
\label{sec:model-lft}

Having established conditions to ensure the closed-loop system is realizable and internally stable, there remains a nonlinear gain block $K$. $K$ can be understood as a time-varying gain dependent on the quantizer input. For example, a 1-bit quantizer (\gls{delta}$ = 2$) with output $\{-1, 1\}$ would have instantaneous gain $K(t) = \frac{1}{u(t)}$. As the value of $u$ at each sample time is not known in advance, $K$ may be modelled as a multiplicative uncertainty. The upper \gls{LFT} allows $K$ to be separated into a constant gain matrix $M_{2 \times 2}$ and a normalized, \gls{Hinf} norm-bounded uncertain block $\Delta$ by Expression~\ref{eq:lft}.

\begin{equation}
	K \leftrightarrow \mathcal{F}_U\{M, \Delta\} \quad ||\Delta||_\infty \leq 1 \label{eq:lft}
\end{equation}

The model from \autoref{fig:sdm-stdf} is shown in \autoref{fig:sdm-stdf-lft} with the quantizer and variable gain replaced by this \gls{LFT} interconnection. In \autoref{ch:Optimization}, it is of interest to ensure the robustness of the system to $\Delta$, which may be achieved using this model.

\begin{figure}[h]
	\centering
	% Augmented Plant
	\begin{tikzpicture}[ampersand replacement=\&,scale=0.75, every node/.style={scale=0.75}]
	
		% Place nodes using a matrix
		\matrix (m1) [row sep=2.5mm, column sep=5mm]
		{
			%-----------------------------------------------------------------------------------------------
			\node[coordinate]					(m00) {};				\&
			\node[coordinate]					(m01) {};				\&
			\node[coordinate]					(m03) {};				\&
			\node[dspsquare]					(m05) {$\Delta$};			\&
			\node[coordinate]					(m06) {};				\& \\
			%-----------------------------------------------------------------------------------------------
			\node[dspnodeopen,dsp/label=left]		(m10) {$r$};				\&
			\node[dspadder,label={below left:$-$}]	(m11) {};				\&
			\node[dspfilter]					(m13) {$H_1(\lambda)$};		\&
			\node[dspfilter,yshift=5]				(m15) {$M$};			\&
			\node[dspnodefull]					(m16) {};				\&
			\node[dspnodeopen,dsp/label=right]		(m17) {$y$};				\& \\
			%-----------------------------------------------------------------------------------------------
			\node[coordinate]					(m20) {};				\&
			\node[coordinate]					(m21) {};				\&
			\node[coordinate]					(m23) {};				\& 
			\node[coordinate]					(m25) {};				\& 
			\node[coordinate]					(m26) {};				\& \\
			%-----------------------------------------------------------------------------------------------
		};
		
		%\node[draw,inner xsep=15pt,inner ysep=10pt,dashed,fit={($(m10) + (0,0.6)$) ($(m16)+(0.85, -1)$)},label={[align=center]below:Augmented Plant $G(\lambda)$}] {};
	
		\draw[dspconn] (m10) to node[midway,above] {$e$} (m11);
		\draw[dspconn] (m11) to node[midway,above] {$u$} (m13);
		\node[coordinate] at ($(m15) + (-20pt,-5pt)$) (m15-sw) {};
		\node[coordinate] at ($(m15) + (20pt,-5pt)$) (m15-se) {};
		\node[coordinate] at ($(m15) + (-20pt,5pt)$) (m15-nw) {};
		\node[coordinate] at ($(m15) + (20pt,5pt)$) (m15-ne) {};
		\node[coordinate] at ($(m15-nw) + (-10pt,0)$) (m15-nw-corner) {};
		\node[coordinate] at ($(m15-ne) + (10pt,0)$) (m15-ne-corner) {};
		\node[coordinate] at (m15-nw-corner |- m05.west) (m05-w-corner) {};
		\node[coordinate] at (m15-ne-corner |- m05.east) (m05-e-corner) {};
		
		\draw[dspconn] (m13) -- (m15-sw);
		\draw[dspconn] (m15-se) -- (m17);
		\draw[dspconn] (m15-nw-corner) -- (m15-nw);
		\draw[dspline] (m15-nw-corner) -- (m05-w-corner);
		\draw[dspline] (m05) -- (m05-w-corner);
		\draw[dspline] (m15-ne-corner) -- (m15-ne);
		\draw[dspline] (m15-ne-corner) -- (m05-e-corner);
		\draw[dspconn] (m05-e-corner) -- (m05);
		\draw[dspline] (m16) -- (m26);
		\draw[dspline] (m26) -- (m21);
		\draw[dspconn] (m21) -- (m11);
		%\draw[black!20, ->, out=-90, in=-90, looseness=0.6] ($(m10)+(-0.35, -0.25)$) to node[below, xshift=-22pt] {$STF$} ($(m17)+(0.35, -0.25)$);
		%\draw[RedOrange, ->, out=90, in=180, looseness=1] ($(m10)+(-0.35, 0.25)$) to node[left, xshift=-2pt,yshift=2pt] {$NTF$} ($(m02)+(-0.35, 0.35)$);
		
	\end{tikzpicture}
	\caption{The linearized block diagram with the quantizer replaced by a multiplicative uncertainty extracted via \glsentryshort{LFT}.} \label{fig:sdm-stdf-lft}
\end{figure}

\section{Derivation of Augmented System}

\subsection{Extraction of Performance and Stability Channels}

Finally, the model is abstracted into an augmented form where all desired input and output channels are present and all unnecessary ones hidden. Let the \gls{LFT} $\Delta \rightarrow M$ input and $M \rightarrow \Delta$ output be \gls{w} and \gls{z}, respectively. These channels are required to be accessed in addition to \gls{r}, \gls{e}, \gls{u}, and \gls{y} for the purposes listed in \autoref{tab:aug-ch}. The augmented system \gls{G} is encapsulated as the dashed block in \autoref{fig:sdm-aug}.

\begin{table}[t]
	\centering
	\caption{Input and output channels of interest for the augmented system.} \label{tab:aug-ch}
	%\renewcommand{\arraystretch}{2}
	\begin{tabular}{>{\centering\arraybackslash}m{1.6cm} | >{\RaggedRight}m{2cm} >{\RaggedRight}m{2.25cm} >{\RaggedRight}m{2.25cm} >{\RaggedRight}m{2cm}}
		\toprule
		\diagbox[width=2cm, height=1cm]{\textbf{Input}}{\textbf{Output}} & \multicolumn{1}{c}{$z$} & \multicolumn{1}{c}{$e$} & \multicolumn{1}{c}{$u$} & \multicolumn{1}{c}{$y$} \\
		\midrule
		$r$ & Not used & \gls{NTF} performance channel & Constraint on quantizer input signal & \gls{STF} constraints for \gls{CT} design \\
		$w$ & Quantizer gain robustness channel & Not used & Not used & Not used \\
		\bottomrule
	\end{tabular}
\end{table}

\begin{figure}
	\centering
	% Augmented Plant
	\begin{tikzpicture}[ampersand replacement=\&,scale=0.75, every node/.style={scale=0.75}]
	
		% Place nodes using a matrix
		\matrix (m1) [row sep=2.5mm, column sep=5mm]
		{
			%-----------------------------------------------------------------------------------------------
			\node[coordinate]					(m00) {};				\&
			\node[coordinate]					(m01) {};				\&
			\node[dspnodeopen,dsp/label=above]		(m02) {$e$};				\&
			\node[coordinate]					(m03) {};				\&
			\node[dspnodeopen,dsp/label=above]		(m04) {$u$};				\&
			\node[dspsquare]					(m05) {$\Delta$};			\&
			\node[coordinate]					(m06) {};				\& \\
			%-----------------------------------------------------------------------------------------------
			\node[dspnodeopen,dsp/label=left]		(m10) {$r$};				\&
			\node[dspadder,label={below left:$-$}]	(m11) {};				\&
			\node[dspnodefull]					(m12) {};				\&
			\node[dspfilter]					(m13) {$H_1(\lambda)$};		\&
			\node[dspnodefull]					(m14) {};				\&
			\node[dspfilter,yshift=5]				(m15) {$M$};			\&
			\node[dspnodefull]					(m16) {};				\&
			\node[dspnodeopen,dsp/label=right]		(m17) {$y$};				\& \\
			%-----------------------------------------------------------------------------------------------
			\node[coordinate]					(m20) {};				\&
			\node[coordinate]					(m21) {};				\&
			\node[coordinate]					(m22) {};				\& 
			\node[coordinate]					(m23) {};				\& 
			\node[coordinate]					(m24) {};				\&
			\node[coordinate]					(m25) {};				\& 
			\node[coordinate]					(m26) {};				\& \\
			%-----------------------------------------------------------------------------------------------
		};
		
		\node[draw,inner xsep=15pt,inner ysep=10pt,dashed,fit={($(m10) + (0,0.6)$) ($(m16)+(0.85, -1)$)},label={[align=center]below:Augmented System $G(\lambda)$}] {};
	
		\draw[dspconn] (m10) -- (m11);
		\draw[dspconn] (m11) -- (m13);
		\draw[dspconn] (m12) -- (m02);
		\node[coordinate] at ($(m15) + (-20pt,-5pt)$) (m15-sw) {};
		\node[coordinate] at ($(m15) + (20pt,-5pt)$) (m15-se) {};
		\node[coordinate] at ($(m15) + (-20pt,5pt)$) (m15-nw) {};
		\node[coordinate] at ($(m15) + (20pt,5pt)$) (m15-ne) {};
		\node[coordinate] at ($(m15-nw) + (-10pt,0)$) (m15-nw-corner) {};
		\node[coordinate] at ($(m15-ne) + (10pt,0)$) (m15-ne-corner) {};
		\node[coordinate] at (m15-nw-corner |- m05.west) (m05-w-corner) {};
		\node[coordinate] at (m15-ne-corner |- m05.east) (m05-e-corner) {};
		
		\draw[dspconn] (m13) -- (m15-sw);
		\draw[dspconn] (m15-se) -- (m17);
		\draw[dspconn] (m15-nw-corner) -- (m15-nw);
		\draw[dspline] (m15-nw-corner) -- (m05-w-corner);
		\draw[dspline] (m05) -- node[midway,above] {$w$} (m05-w-corner);
		\draw[dspline] (m15-ne-corner) -- (m15-ne);
		\draw[dspline] (m15-ne-corner) -- (m05-e-corner);
		\draw[dspconn] (m05-e-corner) -- node[midway,above] {$z$} (m05);
		\draw[dspline] (m16) -- (m26);
		\draw[dspline] (m26) -- (m21);
		\draw[dspconn] (m21) -- (m11);
		\draw[dspconn] (m14) -- (m04);
		%\draw[black!20, ->, out=-90, in=-90, looseness=0.6] ($(m10)+(-0.35, -0.25)$) to node[below, xshift=-22pt] {$STF$} ($(m17)+(0.35, -0.25)$);
		%\draw[RedOrange, ->, out=90, in=180, looseness=1] ($(m10)+(-0.35, 0.25)$) to node[left, xshift=-2pt,yshift=2pt] {$NTF$} ($(m02)+(-0.35, 0.35)$);
		
	\end{tikzpicture}
	\caption{The augmented plant is derived by setting $H_0(\lambda) = 1$, taking the LFT of the uncertain gain, extracting the signals of interest, and writing the closed-loop equations.} \label{fig:sdm-aug}
\end{figure}

\subsection{Derivation of State-Space Model}

Now that the desired input and output signals are captured by the model, it is a simple exercise to write the system in state-space form. To begin, let filter $H_1(\lambda)$ be the transfer function of order \gls{order} in variable \gls{sorz}$ = z$ in the \gls{DT} case (or \gls{sorz}$ = s$ in the \gls{CT} case). The numerator and denominator coefficients are shown in \autoref{eq:h1} which has the equivalent state-space representation of \autoref{eq:h1-ss}.

\begin{align}
	H_1(\lambda) = \frac{U(\lambda)}{E(\lambda)} &= \frac{b_{n-1}\lambda^{n-1} + b_{n-2}\lambda^{n-2} + \ldots + b_1\lambda + b_0}{\lambda^n + a_{n-1}\lambda^{n-1} + a_{n-2}\lambda^{n-2} + \ldots + a_1z\lambda + a_0} \label{eq:h1} \\
	&= C_H(\lambda I - A_H)^{-1}B_H \label{eq:h1-ss}
\end{align}

Naturally, $H_1(\lambda)$ is a strictly proper transfer function and state-space feedthrough matrix $D_H = 0$ due to the constraints proposed in \autoref{sec:model-wp-is}. The constant gain matrix \gls{M} may be split into its constituent parts:

\begin{equation}
	M = 
	\begin{bmatrix}
		m_{11} & m_{12} \\
		m_{21} & m_{22}
	\end{bmatrix}. \label{eq:m-exp}
\end{equation}

With some algebra, the augmented system \gls{G} from \autoref{fig:sdm-aug} may be written in state-space form with notation from Equations~\ref{eq:h1-ss} and \ref{eq:m-exp} by introducing \gls{order}$\times 1$ state vector $x$. The notation $G_{qp}(\lambda)$ is used to indicate the \gls{SISO} transfer function of \gls{G} from some input channel $p$ to some output channel $q$. The closed-loop state-space matrix blocks are denoted with cursive symbols as shown in \autoref{eq:aug-ss}.

\begin{align}
	G:
	\begin{bmatrix}
		\dot{x} \\
		z \\
		e \\
		u \\
		y
	\end{bmatrix} &=
	\begin{bmatrix}[c|cc]
		A_H - m_{22}B_HC_H & -m_{21}B_H & B_H \\
		\cmidrule(lr){1-3}
		m_{12}C_H & m_{11} & 0 \\
		-m_{22}C_H & -m_{21} & 1 \\
		C_H & 0 & 0 \\
		m_{22}C_H & m_{21} & 0
	\end{bmatrix}
	\begin{bmatrix}
		x \\
		w \\
		r
	\end{bmatrix} \label{eq:aug} \\
	&=
	\begin{bmatrix}[c|cc]
		\mathcal{A} & \mathcal{B}_w & \mathcal{B}_r \\
		\cmidrule(lr){1-3}
		\mathcal{C}_z & \mathcal{D}_{zw} & \mathcal{D}_{zr} \\
		\mathcal{C}_e & \mathcal{D}_{ew} & \mathcal{D}_{er} \\
		\mathcal{C}_u & \mathcal{D}_{uw} & \mathcal{D}_{ur} \\
		\mathcal{C}_y & \mathcal{D}_{yw} & \mathcal{D}_{yr}
	\end{bmatrix}
	\begin{bmatrix}
		x \\
		w \\
		r
	\end{bmatrix} \label{eq:aug-ss}
\end{align}

With the channels of interest exposed and the system in a state-space form, one can express design goals as constraints on these channels. In \autoref{ch:Stability}, various stability measures and performance goals are discussed from which those that are ideal from an optimization perspective are selected. In \autoref{ch:Optimization}, the framework is introduced to allow these goals to be applied to the augmented system in a way that allows the optimization problem to be efficiently solved.