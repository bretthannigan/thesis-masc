%% The following is a directive for TeXShop to indicate the main file
%%!TEX root = diss.tex

\chapter{Introduction}
\label{ch:Introduction}

The conversion of signals between analog and digital domanis is an often encountered problem in signal processing. For an analog signal to be represented digitally, it must undergo the processes of sampling and quantization (\autoref{fig:samp-quant}). The former is the conversion from \gls{CT} to \gls{DT} and can be done without loss of information by the Nyquist-Shannon sampling theorem, given a sufficiently high sample rate. The latter is the mapping from an infinite set of possible values to a finite number of quantization levels. Unlike sampling, the process of quantization is non-injective and thus irreversible. The design of signal conversion circuits that minimize the error introduced by quantization is a major problem in mixed signal electronics.

\begin{figure}
	\centering
	\begin{tikzpicture}[ampersand replacement=\&,scale=0.75, every node/.style={scale=0.75}]
	
		\node[coordinate] (w) at (0,0) {};
		\node[coordinate] (n) at (5,1.5) {};
		\node[coordinate] (s) at (5,-1.5) {};
		\node[coordinate] (e) at (8.5,0) {};
		\node[dspsquare] (sh2) at (4,0) {S/H};
		\node[dspsquare] (q2) at (6,0) {\RaisingEdge};
		\node[coordinate] (w-up) at ($(w) + (1,0.5)$) {};
		\node[coordinate] (w-right) at ($(w) + (1.25,0)$) {};
		\node[coordinate] (w-low) at ($(w) + (1,-0.5)$) {};
		\node[coordinate] (n-left) at ($(n) + (-1.5,0)$) {};
		\node[coordinate] (s-left) at  ($(s) + (-1.5,0)$) {};
		\node[coordinate] (n-right) at ($(n) + (1.25,0)$) {};
		\node[coordinate] (s-right) at ($(s) + (1.25,0)$) {};
		\node[coordinate] (mid) at ($(e) + (-1.5,0)$) {};
		\node[coordinate] (e-left) at ($(e) + (-1.5,0)$) {};
		
		\path (w-up) -- (n-left) node[dspsquare,midway,sloped] (sh1) {S/H};
		\path (w-up) -- (n-left) node[midway,sloped,yshift=20] {Sampling};
		\path (w-low) -- (s-left) node[dspsquare,midway,sloped] (q1) {\RaisingEdge};
		\path (w-low) -- (s-left) node[midway,sloped,yshift=-20] {Quantization};
		
		\draw[conndspconn] (w-up) -- (sh1);
		\draw[dspconn] (w-low) -- (q1);
		\draw[conndspconn] (sh1) -- (n-left);
		\draw[dspconn] (q1) -- (s-left);		
		\draw[conndspconn] (w-right) -- (sh2);
		\draw[conndspconn] (sh2) -- (q2);
		\draw[dspconn] (q2) -- (mid);
		
		% Analog signal.
		\begin{axis}[
			at={(w)},
			anchor=center,
			grid=major, 
			x=14mm, y=7mm, 
			axis x line=center, axis y line*=left, 
			ymin=-1, ymax=1, 
			yticklabels={}, xticklabels={},
			xlabel={$t$}, ylabel={$r(t)$},
			every axis x label/.style={
				at={(ticklabel* cs:1.05)},
				anchor=west
				},
			every axis y label/.style={
				at={(ticklabel* cs:0.5)},
				anchor=south,
				rotate=90
				},
			]
		\addplot[domain=-0:1,  smooth,  black]
		  plot {sin(deg(2*pi*1.33*x))};
		\end{axis}
		
		% Sampled plot.
		\begin{axis}[
			at={(n)},
			anchor=center,
			grid=major, 
			x=14mm, y=7mm, 
			axis x line=center, axis y line*=left, 
			ymin=-1, ymax=1, 
			yticklabels={}, xticklabels={}, 
			xlabel={$k$}, ylabel={$r[kT_s]$},
			every axis x label/.style={
				at={(ticklabel* cs:1.05)},
				anchor=west
				},
			every axis y label/.style={
				at={(ticklabel* cs:0.5)},
				anchor=south,
				rotate=90
				},
			]
		\addplot[ycomb, domain=-0:1,  samples=11, solid,  black, mark=*, mark size=1.5pt] 
		  plot {sin(deg(2*pi*1.33*x))};
		\end{axis}
		
		% Quantized plot.
		\begin{axis}[
			at={(s)},
			anchor=center,
			grid=major, 
			x=14mm, y=7mm, 
			axis x line=center, axis y line*=left, 
			ymin=-1, ymax=1, 
			yticklabels={}, xticklabels={}, 
			xlabel={$t$}, ylabel={$r_q(t)$},
			every axis x label/.style={
				at={(ticklabel* cs:1.05)},
				anchor=west
				},
			every axis y label/.style={
				at={(ticklabel* cs:0.5)},
				anchor=south,
				rotate=90
				},
			]
		\addplot[domain=-0:1, black]
		 coordinates {(0,0) (0.030,0) (0.030,0.5) (0.101,0.5) (0.101,1) (0.274,1) (0.274,0.5) (0.345,0.5) (0.345,0) (0.406,0) (0.406,-0.5) (0.477,-0.5) (0.477,-1) (0.650,-1) (0.650,-0.5) (0.721,-0.5) (0.721,0) (0.782,0) (0.782,0.5) (0.853,0.5) (0.853,1) (1,1)};
		 \end{axis}
		
		% Sampled and quantized plot.
		\begin{axis}[
			at={(e)},
			anchor=center,
			grid=major, 
			x=14mm, y=7mm, 
			axis x line=center, axis y line*=left, 
			ymin=-1, ymax=1, 
			yticklabels={}, xticklabels={}, 
			xlabel={$k$}, ylabel={$r_q[kT_s]$},
			every axis x label/.style={
				at={(ticklabel* cs:1.05)},
				anchor=west
				},
			every axis y label/.style={
				at={(ticklabel* cs:0.5)},
				anchor=south,
				rotate=90
				},
			]
		\addplot[ycomb, domain=-0:1, solid, black, mark=*, mark size=1.5pt]
		 coordinates {(0,0) (0.1,0.5) (0.2,1) (0.3,0.5) (0.4,0) (0.5,-1) (0.6,-1) (0.7,-0.5) (0.8,0.5) (0.9,1) (1,1)};
		\end{axis}
	\end{tikzpicture}
	\caption{A continuous-time, continuous-value signal $r(t)$ is sampled to produce a discrete-time, continuous-value signal $r[kT_s]$. $r(t)$ independently undergoes quantization to yield a continuous-time, discrete-value signal $r_q(t)$. When both processes are applied in sequence, a discrete-time, discrete-value signal $r_q[kT_s]$ is the result.}
	\label{fig:samp-quant}
\end{figure}

Sigma delta modulation is a widely used technique for \gls{A/D} and \gls{D/A} conversion of signals that provides high resolution through the techniques of oversampling and noise shaping. Oversampling trades throughput for resolution, thus the sigma delta modulator generally lies between integrating converters, which are specialized for near-dc signals, and high-speed architectures, such as successive approximation and flash. The sigma delta quantization scheme is especially applicable to signals with low to moderate frequency content. Signals with these properties include most biosignals such as those recorded electrically (\gls{EEG}, \gls{ECG}) or through other means using transducers (\gls{PPG}), as well as audio signals.

\section{Oversampling and Noise Shaping}
\label{sec:in-os-ns}

Oversampling is simply the process where the analog signal is sampled at a rate higher than what the sampling theorem would dictate for perfect reconstruction, expressed as the \gls{OSR} relative to the Nyquist frequency. It may seem that this does not have a direct benefit \emph{per se}, but it allows a less demanding analog \gls{AAF} to be used, saving circuit area. It also permits the quantization error to be spread across a larger bandwidth to increase resolution. Assuming quantization error can be modelled by white noise, oversampling reduces the in-band quantization noise power \gls{pq} by a factor directly proportional to \gls{OSR} \cite{DeLaRosa2011} as seen in \autoref{eq:pq-osr}, where \gls{delta} is the difference between quantization levels. These two advantages --- reducing analog circuit complexity and increasing resolution --- are common goals in sigma delta modulator design.

\begin{equation}
	P_Q = \frac{\Delta^2}{12 \cdot OSR} \label{eq:pq-osr}
\end{equation}

It may appear that oversampling alone quickly becomes impractical because one must approach very high sampling frequencies to increase the \gls{SQNR} substantially. However, this assumes that the quantization noise is evenly distributed across the spectrum. Noise shaping is the use of a filter operating on the oversampled signal to push quantization noise out of the signal band where it can be removed by \gls{DRF}. This behaviour is implemented by wrapping the filter and quantizer in a feedback loop. With the same white noise assumption, the tradeoff between in-band shaped quantization noise and \gls{OSR} is improved for ideal loop filters when order \gls{order} is increased as shown in \autoref{eq:pq-osr-ns} \cite{DeLaRosa2011}. The effect of oversampling and noise shaping is demonstrated in \autoref{fig:os-ns-compare}. 

\begin{equation}
	P_Q = \frac{\Delta^2\pi^{2n}}{12\left(2n + 1\right) \cdot OSR^{2n + 1}} \label{eq:pq-osr-ns}
\end{equation}

\begin{figure}
\noindent\makebox[\textwidth]{
	\centering
	\begin{tikzpicture}[ampersand replacement=\&,scale=0.75, every node/.style={scale=0.75}]
		\node[coordinate] (g1) at (-2,2.5) {};
		% Place nodes using a matrix
		\matrix (m0) at (0,5) [row sep=0mm, column sep=5mm, matrix anchor=north west]
		{
			%--------------------------------------------------------------------------------------------------------------------------------
			\node[coordinate]						(m0-00) {};							\&
			\node[coordinate]						(m0-01) {};							\&
			\node[coordinate,label={above:$f_s$}]		(m0-02) {};							\& \\
			%--------------------------------------------------------------------------------------------------------------------------------
			\node[coordinate]						(m0-10) {};							\&
			\node[dspsquare,label={above:AAF}]			(m0-11) {};							\&
			\node[dspsquare]						(m0-12) {S/H};						\&
			\node[dspsquare]						(m0-13) {\RaisingEdge};					\& \\
			%--------------------------------------------------------------------------------------------------------------------------------
		};
		
		\draw[->] (m0-02) -- (m0-12);
		\draw[dspconn] (m0-10) -- (m0-11);
		\draw[dspconn] (m0-11) -- (m0-12);
		\draw[dspconn] (m0-12) -- (m0-13);
		\pgfplotsset{width=2.25cm,height=2.25cm,xmin=0.1,xmax=1000000,ymin=0.0001,ymax=10}
		\begin{loglogaxis}[
				at={($(m0-11) + (-9,-9)$)},
				ticks=none,
				axis x line*=bottom,
				axis y line*=left
			 ]
			\addplot[domain=1:100000]  {(60*x+10000)/(x*x + 60*x+10000)};
		\end{loglogaxis}
		
		\matrix (m1) at (0,2.5) [row sep=0mm, column sep=5mm, matrix anchor=north west]
		{
			%--------------------------------------------------------------------------------------------------------------------------------
			\node[coordinate]						(m1-00) {};							\&
			\node[coordinate]						(m1-01) {};							\&
			\node[coordinate,label={above:$OSR\cdot f_s$}]	(m1-02) {};							\& \\
			%--------------------------------------------------------------------------------------------------------------------------------
			\node[dspnodefull]						(m1-10) {};							\&
			\node[dspsquare,label={above:AAF}]			(m1-11) {};							\&
			\node[dspsquare]						(m1-12) {S/H};						\&
			\node[dspsquare]						(m1-13) {\RaisingEdge};					\&
			\node[dspsquare,label={above:DRF}]			(m1-14) {};							\&
			\node[dspfilter]						(m1-15) {$\downarrow OSR$};				\& \\
			%--------------------------------------------------------------------------------------------------------------------------------
		};
		
		\draw[->] (m1-02) -- (m1-12);
		\draw[dspconn] (m1-10) -- (m1-11);
		\draw[dspconn] (m1-11) -- (m1-12);
		\draw[dspconn] (m1-12) -- (m1-13);
		\draw[dspconn] (m1-13) -- (m1-14);
		\draw[dspconn] (m1-14) -- (m1-15);
		\pgfplotsset{width=2.25cm,height=2.25cm,xmin=0.1,xmax=1000000,ymin=0.0001,ymax=10}
		\begin{loglogaxis}[
				at={($(m1-11) + (-9,-9)$)},
				ticks=none,
				axis x line*=bottom,
				axis y line*=left
			 ]
			\addplot[domain=1:100000]  {(60*x+10000)/(x*x + 60*x+10000)};
		\end{loglogaxis}
		\pgfplotsset{width=2.25cm,height=2.25cm,xmin=0.1,xmax=1000000,ymin=0.0001,ymax=10}
		\begin{loglogaxis}[
				at={($(m1-14) + (-9,-9)$)},
				ticks=none,
				axis x line*=bottom,
				axis y line*=left
			 ]
			\addplot[domain=1:100000]  {(60*x+10000)/(x*x + 60*x+10000)};
		\end{loglogaxis}
		
		\matrix (m2) at (0,0) [row sep=0mm, column sep=5mm, matrix anchor=north west]
		{
			%--------------------------------------------------------------------------------------------------------------------------------
			\node[coordinate]						(m2-00) {};							\&
			\node[coordinate]						(m2-01) {};							\&
			\node[coordinate,label={above:$OSR\cdot f_s$}]	(m2-02) {};							\& \\
			%--------------------------------------------------------------------------------------------------------------------------------
			\node[coordinate]						(m2-10) {};							\&
			\node[dspsquare,label={above:AAF}]			(m2-11) {};							\&
			\node[dspsquare]						(m2-12) {S/H};						\&
			\node[dspadder,label={below left:$-$}]		(m2-13) {};							\&
			\node[dspsquare,label={above:LF}]			(m2-14) {$\int$};						\&
			\node[dspsquare]						(m2-15) {\RaisingEdge};					\&
			\node[dspnodefull]						(m2-16) {};							\&
			\node[dspsquare,label={above:DRF}]			(m2-17) {};							\&
			\node[dspfilter]						(m2-18) {$\downarrow OSR$};				\&
			\node[dspnodeopen,dsp/label=right]			(m2-19) {$y_3[k]$};						\& \\
			%--------------------------------------------------------------------------------------------------------------------------------
			\node[coordinate]						(m2-20) {};							\&
			\node[coordinate]						(m2-21) {};							\&
			\node[coordinate]						(m2-22) {};							\&
			\node[coordinate]						(m2-23) {};							\&
			\node[coordinate]						(m2-24) {};							\&
			\node[coordinate]						(m2-25) {};							\&
			\node[coordinate]						(m2-26) {};							\& \\
		};
		
		\draw[->] (m2-02) -- (m2-12);
		\draw[dspconn] (m2-10) -- (m2-11);
		\draw[dspconn] (m2-11) -- (m2-12);
		\draw[dspconn] (m2-12) -- (m2-13);
		\draw[dspconn] (m2-13) -- (m2-14);
		\draw[dspconn] (m2-14) -- (m2-15);
		\draw[dspline] (m2-15) -- (m2-16);
		\draw[dspconn] (m2-16) -- (m2-17); 
		\draw[dspconn] (m2-17) -- (m2-18);
		\draw[dspconn] (m2-18) -- (m2-19);
		\draw[dspline] (m2-16) -- (m2-26);
		\draw[dspline] (m2-26) -- (m2-23);
		\draw[dspconn] (m2-23) -- (m2-13);
		\pgfplotsset{width=2.25cm,height=2.25cm,xmin=0.1,xmax=1000000,ymin=0.0001,ymax=10}
		\begin{loglogaxis}[
				at={($(m2-11) + (-9,-9)$)},
				ticks=none,
				axis x line*=bottom,
				axis y line*=left
			 ]
			\addplot[domain=1:100000]  {(60*x+10000)/(x*x + 60*x+10000)};
		\end{loglogaxis}
		\pgfplotsset{width=2.25cm,height=2.25cm,xmin=0.1,xmax=1000000,ymin=0.0001,ymax=10}
		\begin{loglogaxis}[
				at={($(m2-17) + (-9,-9)$)},
				ticks=none,
				axis x line*=bottom,
				axis y line*=left
			 ]
			\addplot[domain=1:100000]  {(60*x+10000)/(x*x + 60*x+10000)};
		\end{loglogaxis}
		
		\draw[dspline] (m0-10) -- (m1-10);
		\draw[dspline] (m1-10) -- (m2-10);
		\node[dspnodeopen,dsp/label=left] (r) at ($(m1-10) + (-0.5cm, 0)$) {$r(t)$};
		\draw[dspline] (r) -- (m1-10);
		
		\node[coordinate] (g-y2) at ($(m2-19) + (4cm,0)$) {};
		\coordinate (g-y1) at (g-y2 |- m1-15);
		\coordinate (g-y0) at (g-y2 |- m0-13);
		\node[dspnodeopen,dsp/label=right] (y1) at (m2-19 |- m1-15) {$y_2[k]$};
		\node[dspnodeopen,dsp/label=right] (y0) at (m2-19 |- m0-13) {$y_1[k]$};
		
		\draw[dspconn] (m0-13) -- (y0);
		\draw[dspconn] (m1-15) -- (y1);
		
		\begin{axis}[
			at={($(r) + (-3cm,0)$)},
			anchor=center,
			width=6cm, height=3.75cm,
			anchor=center, 
			xmin=0, xmax=4,
			ymin=-100, ymax=300,
			axis x line=bottom, axis y line=left, axis line style={-},
			xticklabels={0,1,2,3,\SI{4}{\second}}, xtick={0,1,2,3,4},
			yticklabels={, 0, \SI{300}{\micro\volt}}, ytick={-100, 0, 300}
			]
			\addplot[solid,black] table [x=t, y=r, col sep=comma] {data/comparison-r.csv};
		\end{axis}
		
		\begin{axis}[
			at={(g-y0)},
			anchor=center,
			width=6cm, height=3.75cm,
			anchor=center, 
			xmin=0, xmax=4,
			ymin=-100, ymax=300,
			axis x line=none, axis y line=left, axis line style={-},
			xticklabels={0,1,2,3,\SI{4}{\second}}, xtick={0,1,2,3,4},
			yticklabels={, 0, \SI{300}{\micro\volt}}, ytick={-100, 0, 300}
			]
			\addplot[solid,black] table [x=t, y=y, col sep=comma] {data/comparison-y0.csv};
		\end{axis}
		
		\begin{axis}[
			at={(g-y1)},
			anchor=center,
			width=6cm, height=3.75cm,
			anchor=center, 
			xmin=0, xmax=4,
			ymin=-100, ymax=300,
			axis x line=none, axis y line=left, axis line style={-},
			xticklabels={0,1,2,3,\SI{4}{\second}}, xtick={0,1,2,3,4},
			yticklabels={, 0, \SI{300}{\micro\volt}}, ytick={-100, 0, 300}
			]
			\addplot[solid,black] table [x=t, y=y, col sep=comma] {data/comparison-y1.csv};
		\end{axis}
		
		\begin{axis}[
			at={(g-y2)},
			anchor=center,
			width=6cm, height=3.75cm,
			anchor=center, 
			xmin=0, xmax=4,
			ymin=-100, ymax=300,
			axis x line=bottom, axis y line=left, axis line style={-},
			xticklabels={0,1,2,3,\SI{4}{\second}}, xtick={0,1,2,3,4},
			yticklabels={, 0, \SI{300}{\micro\volt}}, ytick={-100, 0, 300}
			]
			\addplot[solid,black] table [x=t, y=y, col sep=comma] {data/comparison-y2.csv};
		\end{axis}
		
	\end{tikzpicture}}
	\caption{A comparison between na\"{i}ve quantization (top), 10 times oversampled quantization (middle), and first order sigma delta modulation (bottom). The graphs on the right show the increasing quality of an \glsentryshort{EEG} signal \cite{Blankertz2007} sampled to a final rate of \SI{100}{\hertz} and quantized with 5 bits by each scheme.}
	\label{fig:os-ns-compare}
\end{figure}

\section{Basic Structure}
\label{sec:in-bs}

We introduce the basic block diagram of a sigma delta modulator and nomenclature that will be used herein. For brevity, we limit the scope to sigma delta \gls{A/D} converters but the concepts are easily transferrable to the \gls{D/A} case. Modulators can be one of two main classes, \gls{CT} or \gls{DT} referring to the nature of the \gls{LF}. 

\subsection{Discrete-Time Modulator}
\label{sec:in-dtm}

\begin{figure}[h]
\noindent\makebox[\textwidth]{
	\centering
	% DT Sigma Delta Modulator
	\begin{tikzpicture}[ampersand replacement=\&,scale=0.75, every node/.style={scale=0.75}]
	
		% Place nodes using a matrix
		\matrix (m1) [row sep=2.5mm, column sep=5mm]
		{
			\node[coordinate]						(m0-00) {};				\&
			\node[coordinate]						(m0-01) {};				\&
			\node[coordinate,label={above:$OSR\cdot f_s$}]	(m0-02) {};				\& \\
			%--------------------------------------------------------------------------------------------------------
			\node[dspnodeopen,dsp/label=left]			(m1-00) {$r_0(t)$};			\&
			\node[dspsquare,label={above:AAF}]			(m1-01) {};				\&
			\node[dspsquare]						(m1-02) {S/H};			\&
			\node[coordinate,label={above:$r[k]$}]		(m1-03) {};				\&
			\node[dspfilter]						(m1-04) {$H_0(z)$};		\&
			\node[coordinate]						(m1-05) {};				\&
			\node[coordinate]						(m1-06) {};				\&
			\node[dspnodeopen,dsp/label=above]			(m1-07) {$d[k]$};			\& \\
			%--------------------------------------------------------------------------------------------------------
			\node[coordinate]						(m2-00) {};				\&
			\node[coordinate]						(m2-01) {};				\&
			\node[coordinate]						(m2-02) {};				\&
			\node[coordinate]						(m2-03) {};				\&
			\node[dspadder,label={below left:$-$}]		(m2-04) {};				\&
			\node[dspfilter]						(m2-05) {$H_1(z)$};		\&
			\node[coordinate,label={$u[k]$}]			(m2-06) {};				\&
			\node[dspsquare,label={below:Quantizer}]		(m2-07) {\RaisingEdge};		\&
			\node[dspnodefull]						(m2-08) {};				\&
			\node[coordinate,label={above:$y[k]$}]		(m2-09) {};				\&
			\node[dspsquare,label={above:DRF}]			(m2-10) {};				\& 
			\node[dspfilter]						(m2-11) {$\downarrow OSR$};	\&
			\node[dspnodeopen,dsp/label=right]			(m2-12) {$y[m]$};		\& \\
			%--------------------------------------------------------------------------------------------------------
			\node[coordinate]						(m3-00) {};				\&
			\node[coordinate]						(m3-01) {};				\&
			\node[coordinate]						(m3-02) {};				\&
			\node[coordinate]						(m3-03) {};				\&
			\node[coordinate]						(m3-04) {};				\& 
			\node[coordinate]						(m3-05) {};				\& 
			\node[coordinate]						(m3-06) {};				\& 
			\node[coordinate]						(m3-07) {};				\& 
			\node[coordinate]						(m3-08) {};				\& 
			\node[coordinate]						(m3-09) {};				\& \\
			%--------------------------------------------------------------------------------------------------------
		};
	
		\node[draw,inner xsep=15pt,inner ysep=10pt,dashed,fit={($(m1-07.north)+(-0.5, 0.7)$) ($(m2-07.south)+(0.5, -0.6)$)},label={[align=center]above:Linear Model}] {};
		\node[draw,inner xsep=15pt,inner ysep=10pt,dashed,fit={($(m1-04.north west)+(-0.5, 0.3)$) ($(m2-05.south east)+(0.4, -0.5)$)},label=above:{Loop Filter $H(z)$}] {};

		\draw[->]		(m0-02) -- (m1-02);
		\draw[dspconn]	(m1-00) -- (m1-01);
		\draw[dspconn]	(m1-01) to node[midway,above] {$r(t)$} (m1-02);
		\draw[dspconn] 	(m1-02) -- (m1-04);
		\draw[dspline] 	(m1-04) -- (m2-04);
		\draw[dspline] 	(m2-05) -- (m2-07);
		\draw[dspconn] 	($(m2-06.east)+(6pt, 0)$) -- (m2-07);
		\draw[dspline]	(m2-07) -- (m2-08);
		\draw[dspconn] 	(m2-08) -- (m2-10);
		\draw[dspline] 	(m2-08) -- (m3-08);
		\draw[dspline] 	(m3-08) -- (m3-03);
		\draw[dspline]	(m3-03) -- (m2-03);
		\draw[dspconn] 	(m2-03) -- (m2-04);
		\draw[dspconn] 	(m2-04) to node[midway,above] {$e[k]$} (m2-05);
		\draw[dspconn] 	(m1-07) -- (m2-07);
		\draw[dspconn]	(m2-10) -- (m2-11);
		\draw[dspconn]	(m2-11) -- (m2-12);
		\draw[OliveGreen, ->, out=40, in=90, looseness=0.85] ($(m1-03)+(0, 0.8)$) to node[below, xshift=-22pt] {$T(z)$} ($(m2-09)+(0.1, 0.8)$);
		\draw[RedOrange, ->, out=-45, in=135, looseness=1] ($(m1-07)+(0.35, 0.25)$) to node[above, xshift=5pt] {$S(z)$} ($(m2-09)+(0, 0.75)$);
		
		\pgfplotsset{width=2.25cm,height=2.25cm,xmin=0.1,xmax=1000000,ymin=0.0001,ymax=10}
		\begin{loglogaxis}[
				at={($(m1-01) + (-9,-9)$)},
				ticks=none,
				axis x line*=bottom,
				axis y line*=left
			 ]
			\addplot[domain=1:100000]  {(60*x+10000)/(x*x + 60*x+10000)};
		\end{loglogaxis}
		\pgfplotsset{width=2.25cm,height=2.25cm,xmin=0.1,xmax=1000000,ymin=0.0001,ymax=10}
		\begin{loglogaxis}[
				at={($(m2-10) + (-9,-9)$)},
				ticks=none,
				axis x line*=bottom,
				axis y line*=left
			 ]
			\addplot[domain=1:100000]  {(60*x+10000)/(x*x + 60*x+10000)};
		\end{loglogaxis}
	\end{tikzpicture}}
	\caption{The basic block diagram of a \glsentryshort{DT} sigma delta \glsentryshort{A/D} converter.} \label{fig:basic-struct-dt}
\end{figure}

For the \gls{DT} class of modulators, we reference \autoref{fig:basic-struct-dt}. The analog front-end includes the \gls{AAF} and sample-and-hold block. This subsystem conditions the input signal $r_0(t)$ and samples it outside the loop to produce \gls{DT} signal $r[k]$. In the modulator loop, the 2-input 1-output \gls{LF} operates on $r[k]$ and the feedback signal, producing intermediate signal $u[k]$ with shaped noise. Then, $u[k]$ undergoes quantization, which is modelled as the addition of an error signal $d[k]$ producing quantizer output $y[k]$. The quantizer output is fed back to the \gls{LF} and also passed along. The final subsystem filters the signal from the shaped noise in the digital domain with a downsampling \gls{DRF} to yield the final digital output $y[m]$. 

From a control systems perspective, there are a couple of transfer functions that will be used to analyze and synthesize loop filters. The sensitivity function \gls{S}, where \gls{sorz}$=z$, is known as the \gls{NTF} of the modulator because it shows how the quantization noise is filtered in the linearized model. The complementary sensitivity function \gls{T} is known as the \gls{STF} of the modulator and shows how the signal is transformed by the modulator loop.

\subsection{Continuous-Time Modulator}
\label{sec:in-ctm}

\begin{figure}[h]
\noindent\makebox[\textwidth]{
	\centering
	% CT Sigma Delta Modulator
	\begin{tikzpicture}[ampersand replacement=\&,scale=0.75, every node/.style={scale=0.75}]
	
		% Place nodes using a matrix
		\matrix (m1) [row sep=2.5mm, column sep=5mm]
		{
			\node[dspnodeopen,label={left:$r(t)$}]		(m1-02) {};				\&
			\node[coordinate]						(m1-03) {};				\&
			\node[dspfilter]						(m1-04) {$H_0(s)$};		\&
			\node[coordinate]						(m1-05) {};				\&
			\node[coordinate]						(m1-06) {};				\&
			\node[dspnodeopen,dsp/label=above]			(m1-07) {$d[k]$};			\& \\
			%--------------------------------------------------------------------------------------------------------
			\node[coordinate]						(m2-02) {};				\&
			\node[coordinate]						(m2-03) {};				\&
			\node[dspadder,label={below left:$-$}]		(m2-04) {};				\&
			\node[dspfilter]						(m2-05) {$H_1(s)$};		\&
			\node[dspsquare]						(m2-06) {S/H};			\&
			\node[dspsquare,label={below:Quantizer}]		(m2-07) {\RaisingEdge};		\&
			\node[dspnodefull]						(m2-08) {};				\&
			\node[coordinate,label={above:$y[k]$}]		(m2-09) {};				\&
			\node[dspsquare,label={above:DRF}]			(m2-10) {};				\& 
			\node[dspfilter]						(m2-11) {$\downarrow OSR$};	\&
			\node[dspnodeopen,dsp/label=right]			(m2-12) {$y[m]$};			\& \\
			%--------------------------------------------------------------------------------------------------------
			\node[coordinate]						(m3-02) {};				\&
			\node[coordinate]						(m3-03) {};				\&
			\node[coordinate]						(m3-04) {};				\& 
			\node[coordinate]						(m3-05) {};				\& 
			\node[dspadc]						(m3-06) {$P(s)$};			\& 
			\node[coordinate]						(m3-07) {};				\& 
			\node[coordinate]						(m3-08) {};				\& 
			\node[coordinate]						(m3-09) {};				\& \\
			%--------------------------------------------------------------------------------------------------------
		};
	
		\node[draw,inner xsep=15pt,inner ysep=10pt,dashed,fit={($(m1-07.north)+(-0.5, 0.7)$) ($(m2-07.south)+(0.5, -0.6)$)},label={[align=center]above:Linear Model}] {};
		\node[draw,inner xsep=15pt,inner ysep=10pt,dashed,fit={($(m1-04.north west)+(-0.5, 0.3)$) ($(m2-05.south east)+(0.4, -0.5)$)},label=above:{Loop Filter $H(s)$}] {};

		\draw[dspconn] 	(m1-02) -- (m1-04);
		\draw[dspline] 	(m1-04) -- (m2-04);
		\draw[dspconn] 	(m2-05) -- (m2-06);
		\draw[dspconn]	(m2-06) to node[midway,above,xshift=-7pt] {$u[k]$} (m2-07);
		\draw[dspconn] 	($(m2-06.east)+(6pt, 0)$) -- (m2-07);
		\draw[dspline]	(m2-07) -- (m2-08);
		\draw[dspconn] 	(m2-08) -- (m2-10);
		\draw[dspline] 	(m2-08) -- (m3-08);
		\draw[dspconn] 	(m3-08) -- (m3-06);
		\draw[dspline]	(m3-06) -- (m3-03);
		\draw[dspline]	(m3-03) -- (m2-03);
		\draw[dspconn] 	(m2-03) -- (m2-04);
		\draw[dspconn] 	(m2-04) to node[midway,above] {$e(t)$} (m2-05);
		\draw[dspconn] 	(m1-07) -- (m2-07);
		\draw[dspconn]	(m2-10) -- (m2-11);
		\draw[dspconn]	(m2-11) -- (m2-12);
		\draw[OliveGreen, ->, out=40, in=90, looseness=0.85] ($(m1-03)+(0, 0.8)$) to node[below, xshift=-22pt] {$T$} ($(m2-09)+(0.1, 0.8)$);
		\draw[RedOrange, ->, out=-45, in=135, looseness=1] ($(m1-07)+(0.35, 0.25)$) to node[above, xshift=5pt] {$S$} ($(m2-09)+(0, 0.75)$);
		
		\pgfplotsset{width=2.25cm,height=2.25cm,xmin=0.1,xmax=1000000,ymin=0.0001,ymax=10}
		\begin{loglogaxis}[
				at={($(m2-10) + (-9,-9)$)},
				ticks=none,
				axis x line*=bottom,
				axis y line*=left
			 ]
			\addplot[domain=1:100000]  {(60*x+10000)/(x*x + 60*x+10000)};
		\end{loglogaxis}
	\end{tikzpicture}}
	\caption{The basic block diagram of a \glsentryshort{DT} sigma delta \glsentryshort{A/D} converter.} \label{fig:basic-struct-ct}
\end{figure}

For the \gls{CT} class of modulators, we reference \autoref{fig:basic-struct-ct}. The structure is similar except the \gls{LF} operates directly on analog input $r(t)$ in the \gls{CT} domain and sampling is done inside the loop. The \gls{AAF} is no longer necessary in most cases as the \gls{LF} precedes the sampling block and implicitly attenuates components of the signal that would result in aliasing. Finally, signal $y[k]$ must undergo \gls{D/A} conversion during feedback, modelled witht the pulse transfer function $P(s)$. 

The \gls{NTF} and \gls{STF} of a \gls{CT} sigma delta modulator are more difficult to define because they are transfer functions involving both \gls{CT} and \gls{DT} signals. The \gls{DT} equivalence principle states that there is a \gls{DT} modulator model that exactly describes the \gls{CT} design at the sampling instants, because the modulator is overall a sampled data system \cite[Sec. 3.2]{Ortmanns2005}. Thus, \gls{DT} transfer functions can be derived for this purpose. However, these equivalent transfer functions may be difficult to manipulate due to their dependence on $P(s)$. For the purposes of this thesis, we omit the sampling block during design and use the simplification that \gls{S} and \gls{T} are \gls{CT} (\gls{sorz}$=s$) transfer functions mapping $d(t) \rightarrow y(t)$ and $r(t) \rightarrow y(t)$, respectively.

\section{Loop Filter}
\label{sec:in-lf}

Together, quantization and noise shaping permit a coarser quantizer element to be used. A common design pattern is to use a high ($>2$) order \gls{LF} paired with a 1-bit quantizer, which is advantageous from a circuit design perspective because a quantizer with just two levels is inherently linear. In addition, low order sigma delta loops often suffer from spurious tones \cite[Sec. 2.6.1]{Schreier1997}. Unfortunately, as \gls{LF} order is increased, the tendency of the loop to become unstable does as well. While first and second order designs are provably stable for DC inputs \cite{Hein1993}, high order filters require careful design to avoid instability. Ensuring stability while maintaining performance is a difficult task due to the presence of the highly nonlinear quantizer. Note that the nonlinearity makes analysis complicated, a stable linear model does not imply a stable modulator while an unstable model can even result in a stable modulator \cite{Risbo1994}.

The design of the noise shaping loop filter is the focus of this thesis. Modelling the loop filter as a 2-input 1-output system as shown in \autoref{sec:in-bs} allows the \gls{NTF} to be determined by $H_1(\lambda)$ alone while the \gls{STF} can be modified independently with filter $H_0(\lambda)$, without loss of generality:

\begin{align*}
	S(\lambda) = \frac{1}{1 - H_1(\lambda)} \\
	T(\lambda) = \frac{H_0(\lambda)}{1 - H_1(\lambda)}.
\end{align*}

We desire an \gls{NTF} that results in a stable linear model, rejects noise in the signal band as much as possible, and has low gain in the out-of-band region to promote stability. The \gls{STF} is less important as $H_0(\lambda)$ can be interpreted as a pre-filter to modify the \gls{STF}, but we prefer unity gain in the signal band.

For a first order modulator, a pure integrator can be used as the loop filter $H_0(\lambda)$. For higher orders, it is common to choose a prototype \gls{NTF} from a family of filters. For example, the popular Delta Sigma Toolbox for MATLAB \cite[Appx. B]{Schreier1997} uses a Chebyshev type II filter for this purpose. The choice of filter greatly affects the stability of the loop, so the traditional design procedure involves extensive simulation under varying input conditions to ensure instability is unlikely during normal operation. Once unstable, the filter states must be reset in order to restore operation. Various schemes to detect the onset of instability \cite{Wong2004} and avoid it with gain scaling \cite{Sooch1989}, internal linear feedback \cite{Moussavi1994}, and automatic resetting schemes \cite{Eynde1991}.

\section{Related Works}
\label{sec:in-rw}

Optimization techniques have been used to design \gls{NTF}s with more degrees of freedom than those made with a single filter prototype. A simple example is that from \cite[Sec. 4.3]{Schreier1997}, where the zeros of the prototype \gls{NTF} are optimized by approximating the integral of the \gls{NTF} in the pass-band, then minimizing it analytically by equating its derivative to zero. The procedure results in an optimal spreading of zeros across the signal bandwidth for the given \gls{NTF} poles. One of the first optimization-based approaches to \gls{NTF} design was the \gls{CLANS} methodology that minimizes \gls{pq} under the white quantization noise assumption \cite{Kenney1988}. This is done using nonlinear optimization to find stable \gls{NTF} pole locations that minimize the accumulation of quantization error subject to some stability and realizability constraints.

Using the principles from $\mathcal{H}_\infty$ control and its associated \gls{LMI} methods, one can define the quantizer as a very simple feedthrough plant and introduce weighting filters on the feedback error signal \gls{e}, loop filter output \gls{u}, and quantizer output \gls{y} to design the loop filter as a controller for various performance and stability constraints \cite{Oberoi2004}. However, the system is bound to the order of the plant augmented with weighting filters and relies on the designer to choose the weights. Choosing weighting filters that are ideal is almost as difficult a task as just choosing the prototype \gls{NTF} directly. Despite this, if a known \gls{AAF} or \gls{DRF} is specified in advance, it may be used as a sort of weighting filter and an optimal \gls{LF} can be designed around it \cite{Ohno2017b}. Applications for this method could be optimizing the \gls{STF} to a psychoacoustic model or making use of existing filters in the signal path.

More recently, the \gls{GKYP} lemma has been applied to sigma delta modulator design. The lemma provides a link between a finite frequency domain inequality, such as specifications on the \gls{NTF} gain, and a linear matrix inequality condition, which can be solved using efficient interior point methods. Using this lemma, the techniques of $\mathcal{H}_\infty$ control can be applied to a transfer function but restricted to a frequency band. This eliminates the need for weighting filters that specify a select band of interest. Unfortunately, the problem becomes non-convex and hard to solve if both poles and zeros are to be optimized simultaneously as is the case with an \gls{IIR} filter. As a workaround, the \gls{NTF} poles may be fixed to a prototype design and just the zeros optimized \cite{Osqui2007}, similar to what was described above. Alternatively, a \gls{FIR} \gls{NTF} form may be assumed \cite{Nagahara2012, Tariq2016} then possibly converted to \gls{IIR} form using approximate methods such as least-squares or Yule-Walker \cite{Tariq2017}. Aside from the large delay introduced, the \gls{FIR} form is not the optimal choice according to \cite{Derpich2008}. Iterative methods have shown promise in providing a workaround to the non-convexity associated with direct \gls{IIR} design. A survey of some of these methods is presented in \cite{Callegari2015a}.

\section{Organization of this Thesis}
\label{sec:in-org}


This document provides a quick set of instructions for using the
\class{ubcdiss} class to write a dissertation in \LaTeX. 
Unfortunately this document cannot provide an introduction to using
\LaTeX.  The classic reference for learning \LaTeX\ is
\citeauthor{lamport-1994-ladps}'s
book~\cite{lamport-1994-ladps}.  There are also many freely-available
tutorials online;
\webref{http://www.andy-roberts.net/misc/latex/}{Andy Roberts' online
    \LaTeX\ tutorials}
seems to be excellent.
The source code for this docment, however, is intended to serve as
an example for creating a \LaTeX\ version of your dissertation.

We start by discussing organizational issues, such as splitting
your dissertation into multiple files, in
\autoref{sec:SuggestedThesisOrganization}.
We then cover the ease of managing cross-references in \LaTeX\ in
\autoref{sec:CrossReferences}.
We cover managing and using bibliographies with \BibTeX\ in
\autoref{sec:BibTeX}. 
We briefly describe typesetting attractive tables in
\autoref{sec:TypesettingTables}.
We briefly describe including external figures in
\autoref{sec:Graphics}, and using special characters and symbols
in \autoref{sec:SpecialSymbols}.
As it is often useful to track different versions of your dissertation,
we discuss revision control further in
\autoref{sec:DissertationRevisionControl}. 
We conclude with pointers to additional sources of information in
\autoref{sec:Conclusions}.

%%%%%%%%%%%%%%%%%%%%%%%%%%%%%%%%%%%%%%%%%%%%%%%%%%%%%%%%%%%%%%%%%%%%%%
\section{Suggested Thesis Organization}
\label{sec:SuggestedThesisOrganization}

The specifies a particular arrangement of the
components forming a thesis.\footnote{See
    \url{http://www.grad.ubc.ca/current-students/dissertation-thesis-preparation/order-components}}
This template reflects that arrangement.

In terms of writing your thesis, the recommended best practice for
organizing large documents in \LaTeX\ is to place each chapter in
a separate file.  These chapters are then included from the main
file through the use of \verb+\include{file}+.  A thesis might
be described as six files such as \file{intro.tex},
\file{relwork.tex}, \file{model.tex}, \file{eval.tex},
\file{discuss.tex}, and \file{concl.tex}.

We also encourage you to use macros for separating how something
will be typeset (\eg bold, or italics) from the meaning of that
something. 
For example, if you look at \file{intro.tex}, you will see repeated
uses of a macro \verb+\file{}+ to indicate file names.
The \verb+\file{}+ macro is defined in the file \file{macros.tex}.
The consistent use of \verb+\file{}+ throughout the text not only
indicates that the argument to the macro represents a file (providing
meaning or semantics), but also allows easily changing how
file names are typeset simply by changing the definition of the
\verb+\file{}+ macro.
\file{macros.tex} contains other useful macros for properly typesetting
things like the proper uses of the latinate \emph{exempli grati\={a}}
and \emph{id est} (\ie \verb+\eg+ and \verb+\ie+), 
web references with a footnoted \acs{URL} (\verb+\webref{url}{text}+),
as well as definitions specific to this documentation
(\verb+\latexpackage{}+).

%%%%%%%%%%%%%%%%%%%%%%%%%%%%%%%%%%%%%%%%%%%%%%%%%%%%%%%%%%%%%%%%%%%%%%
\section{Making Cross-References}
\label{sec:CrossReferences}

\LaTeX\ make managing cross-references easy, and the \latexpackage{hyperref}
package's\ \verb+\autoref{}+ command\footnote{%
    The \latexpackage{hyperref} package is included by default in this
    template.}
makes it easier still. 

A thing to be cross-referenced, such as a section, figure, or equation,
is \emph{labelled} using a unique, user-provided identifier, defined
using the \verb+\label{}+ command.  
The thing is referenced elsewhere using the \verb+\autoref{}+ command.
For example, this section was defined using:
\begin{lstlisting}
    \section{Making Cross-References}
    \label{sec:CrossReferences}
\end{lstlisting}
References to this section are made as follows:
\begin{lstlisting}
    We then cover the ease of managing cross-references in \LaTeX\
    in \autoref{sec:CrossReferences}.
\end{lstlisting}
\verb+\autoref{}+ takes care of determining the \emph{type} of the 
thing being referenced, so the example above is rendered as
\begin{quote}
    We then cover the ease of managing cross-references in \LaTeX\
    in \autoref{sec:CrossReferences}.
\end{quote}

The label is any simple sequence of characters, numbers, digits,
and some punctuation marks such as ``:'' and ``--''; there should
be no spaces.  Try to use a consistent key format: this simplifies
remembering how to make references.  This document uses a prefix
to indicate the type of the thing being referenced, such as \texttt{sec}
for sections, \texttt{fig} for figures, \texttt{tbl} for tables,
and \texttt{eqn} for equations.

For details on defining the text used to describe the type
of \emph{thing}, search \file{diss.tex} and the \latexpackage{hyperref}
documentation for \texttt{autorefname}.


%%%%%%%%%%%%%%%%%%%%%%%%%%%%%%%%%%%%%%%%%%%%%%%%%%%%%%%%%%%%%%%%%%%%%%
\section{Managing Bibliographies with \BibTeX}
\label{sec:BibTeX}

One of the primary benefits of using \LaTeX\ is its companion program,
\BibTeX, for managing bibliographies and citations.  Managing
bibliographies has three parts: (i) describing references,
(ii)~citing references, and (iii)~formatting cited references.

\subsection{Describing References}

\BibTeX\ defines a standard format for recording details about a
reference.  These references are recorded in a file with a
\file{.bib} extension.  \BibTeX\ supports a broad range of
references, such as books, articles, items in a conference proceedings,
chapters, technical reports, manuals, dissertations, and unpublished
manuscripts. 
A reference may include attributes such as the authors,
the title, the page numbers, t.  A reference
can also be augmented with personal attributes, such as a rating,
notes, or keywords.

Each reference must be described by a unique \emph{key}.\footnote{%
    Note that the citation keys are different from the reference
    identifiers as described in \autoref{sec:CrossReferences}.}
A key is a simple sequence of characters, numbers, digits, and some
punctuation marks such as ``:'' and ``--''; there should be no spaces. 
A consistent key format simiplifies remembering how to make references. 
For example:
\begin{quote}
   \fbox{\emph{last-name}}\texttt{-}\fbox{\emph{year}}\texttt{-}\fbox{\emph{contracted-title}}
\end{quote}
where \emph{last-name} represents the last name for the first author,
and \emph{contracted-title} is some meaningful contraction of the
title.  Then \citeauthor{kiczales-1997-aop}'s seminal article on
aspect-oriented programming~\cite{kiczales-1997-aop} (published in
\citeyear{kiczales-1997-aop}) might be given the key
\texttt{kiczales-1997-aop}.

An example of a \BibTeX\ \file{.bib} file is included as
\file{biblio.bib}.  A description of the format a \file{.bib}
file is beyond the scope of this document.  We instead encourage
you to use one of the several reference managers that support the
\BibTeX\ format such as
\webref{http://jabref.sourceforge.net}{JabRef} (multiple platforms) or
\webref{http://bibdesk.sourceforge.net}{BibDesk} (MacOS\,X only). 
These front ends are similar to reference manages such as
EndNote or RefWorks.


\subsection{Citing References}

Having described some references, we then need to cite them.  We
do this using a form of the \verb+\cite+ command.  For example:
\begin{lstlisting}
    \citet{kiczales-1997-aop} present examples of crosscutting 
    from programs written in several languages.
\end{lstlisting}
When processed, the \verb+\citet+ will cause the paper's authors
and a standardized reference to the paper to be inserted in the
document, and will also include a formatted citation for the paper
in the bibliography.  For example:
\begin{quote}
    \citet{kiczales-1997-aop} present examples of crosscutting 
    from programs written in several languages.
\end{quote}
There are several forms of the \verb+\cite+ command (provided
by the \latexpackage{natbib} package), as demonstrated in
\autoref{tbl:natbib:cite}.
Note that the form of the citation (numeric or author-year) depends
on the bibliography style (described in the next section).
The \verb+\citet+ variant is used when the author names form
an object in the sentence, whereas the \verb+\citep+ variant
is used for parenthetic references, more like an end-note.
Use \verb+\nocite+ to include a citation in the bibliography
but without an actual reference.
\nocite{rowling-1997-hpps}
\begin{table}
\caption{Available \texttt{cite} variants; the exact citation style
    depends on whether the bibliography style is numeric or author-year.}
\label{tbl:natbib:cite}
\centering
\begin{tabular}{lp{3.25in}}\toprule
Variant & Result \\
\midrule
% We cheat here to simulate the cite/citep/citet for APA-like styles
\verb+\cite+ & Parenthetical citation (\eg ``\cite{kiczales-1997-aop}''
    or ``(\citeauthor{kiczales-1997-aop} \citeyear{kiczales-1997-aop})'') \\
\verb+\citet+ & Textual citation: includes author (\eg
    ``\citet{kiczales-1997-aop}'' or
    or ``\citeauthor{kiczales-1997-aop} (\citeyear{kiczales-1997-aop})'') \\
\verb+\citet*+ & Textual citation with unabbreviated author list \\
\verb+\citealt+ & Like \verb+\citet+ but without parentheses \\
\verb+\citep+ & Parenthetical citation (\eg ``\cite{kiczales-1997-aop}''
    or ``(\citeauthor{kiczales-1997-aop} \citeyear{kiczales-1997-aop})'') \\
\verb+\citep*+ & Parenthetical citation with unabbreviated author list \\
\verb+\citealp+ & Like \verb+\citep+ but without parentheses \\
\verb+\citeauthor+ & Author only (\eg ``\citeauthor{kiczales-1997-aop}'') \\
\verb+\citeauthor*+ & Unabbreviated authors list 
    (\eg ``\citeauthor*{kiczales-1997-aop}'') \\
\verb+\citeyear+ & Year of citation (\eg ``\citeyear{kiczales-1997-aop}'') \\
\bottomrule
\end{tabular}
\end{table}

\subsection{Formatting Cited References}

\BibTeX\ separates the citing of a reference from how the cited
reference is formatted for a bibliography, specified with the
\verb+\bibliographystyle+ command. 
There are many varieties, such as \texttt{plainnat}, \texttt{abbrvnat},
\texttt{unsrtnat}, and \texttt{vancouver}.
This document was formatted with \texttt{abbrvnat}.
Look through your \TeX\ distribution for \file{.bst} files. 
Note that use of some \file{.bst} files do not emit all the information
necessary to properly use \verb+\citet{}+, \verb+\citep{}+,
\verb+\citeyear{}+, and \verb+\citeauthor{}+.

There are also packages available to place citations on a per-chapter
basis (\latexpackage{bibunits}), as footnotes (\latexpackage{footbib}),
and inline (\latexpackage{bibentry}).
Those who wish to exert maximum control over their bibliography
style should see the amazing \latexpackage{custom-bib} package.

%%%%%%%%%%%%%%%%%%%%%%%%%%%%%%%%%%%%%%%%%%%%%%%%%%%%%%%%%%%%%%%%%%%%%%
\section{Typesetting Tables}
\label{sec:TypesettingTables}

\citet{lamport-1994-ladps} made one grievous mistake
in \LaTeX: his suggested manner for typesetting tables produces
typographic abominations.  These suggestions have unfortunately
been replicated in most \LaTeX\ tutorials.  These
abominations are easily avoided simply by ignoring his examples
illustrating the use of horizontal and vertical rules (specifically
the use of \verb+\hline+ and \verb+|+) and using the
\latexpackage{booktabs} package instead.

The \latexpackage{booktabs} package helps produce tables in the form
used by most professionally-edited journals through the use of
three new types of dividing lines, or \emph{rules}.
% There are times that you don't want to use \autoref{}
Tables~\ref{tbl:natbib:cite} and~\ref{tbl:LaTeX:Symbols} are two
examples of tables typeset with the \latexpackage{booktabs} package.
The \latexpackage{booktabs} package provides three new commands
for producing rules:
\verb+\toprule+ for the rule to appear at the top of the table,
\verb+\midrule+ for the middle rule following the table header,
and \verb+\bottomrule+ for the bottom-most at the end of the table.
These rules differ by their weight (thickness) and the spacing before
and after.
A table is typeset in the following manner:
\begin{lstlisting}
    \begin{table}
    \caption{The caption for the table}
    \label{tbl:label}
    \centering
    \begin{tabular}{cc}
    \toprule
    Header & Elements \\
    \midrule
    Row 1 & Row 1 \\
    Row 2 & Row 2 \\
    % ... and on and on ...
    Row N & Row N \\
    \bottomrule
    \end{tabular}
    \end{table}
\end{lstlisting}
See the \latexpackage{booktabs} documentation for advice in dealing with
special cases, such as subheading rules, introducing extra space
for divisions, and interior rules.

%%%%%%%%%%%%%%%%%%%%%%%%%%%%%%%%%%%%%%%%%%%%%%%%%%%%%%%%%%%%%%%%%%%%%%
\section{Figures, Graphics, and Special Characters}
\label{sec:Graphics}

Most \LaTeX\ beginners find figures to be one of the more challenging
topics.  In \LaTeX, a figure is a \emph{floating element}, to be
placed where it best fits.
The user is not expected to concern him/herself with the placement
of the figure.  The figure should instead be labelled, and where
the figure is used, the text should use \verb+\autoref+ to reference
the figure's label.
\autoref{fig:latex-affirmation} is an example of a figure.
\begin{figure}
    \centering
    % For the sake of this example, we'll just use text
    %\includegraphics[width=3in]{file}
    \Huge{\textsf{\LaTeX\ Rocks!}}
    \caption{Proof of \LaTeX's amazing abilities}
    \label{fig:latex-affirmation}   % label should change
\end{figure}
A figure is generally included as follows:
\begin{lstlisting}
    \begin{figure}
    \centering
    \includegraphics[width=3in]{file}
    \caption{A useful caption}
    \label{fig:fig-label}   % label should change
    \end{figure}
\end{lstlisting}
There are three items of note:
\begin{enumerate}
\item External files are included using the \verb+\includegraphics+
    command.  This command is defined by the \latexpackage{graphicx} package
    and can often natively import graphics from a variety of formats.
    The set of formats supported depends on your \TeX\ command processor.
    Both \texttt{pdflatex} and \texttt{xelatex}, for example, can
    import \textsc{gif}, \textsc{jpg}, and \textsc{pdf}.  The plain
    version of \texttt{latex} only supports \textsc{eps} files.

\item The \verb+\caption+ provides a caption to the figure. 
    This caption is normally listed in the List of Figures; you
    can provide an alternative caption for the LoF by providing
    an optional argument to the \verb+\caption+ like so:
    \begin{lstlisting}
    \caption[nice shortened caption for LoF]{%
	longer detailed caption used for the figure}
    \end{lstlisting}
    \ac{GPS} generally prefers shortened single-line captions
    in the LoF: multiple-line captions are a bit unwieldy.

\item The \verb+\label+ command provides for associating a unique, user-defined,
    and descriptive identifier to the figure.  The figure can be
    can be referenced elsewhere in the text with this identifier
    as described in \autoref{sec:CrossReferences}.
\end{enumerate}
See Keith Reckdahl’s excellent guide for more details,
\webref{http://www.ctan.org/tex-archive/info/epslatex.pdf}{\emph{Using
imported graphics in LaTeX2e}}.

\section{Special Characters and Symbols}
\label{sec:SpecialSymbols}

\LaTeX\ appropriates many common symbols for its own purposes,
with some used for commands (\ie \verb+\{}&%+) and
mathematics (\ie \verb+$^_+), and others are automagically transformed
into typographically-preferred forms (\ie \verb+-`'+) or to
completely different forms (\ie \verb+<>+).
\autoref{tbl:LaTeX:Symbols} presents a list of common symbols and
their corresponding \LaTeX\ commands.  A much more comprehensive list 
of symbols and accented characters is available at:
\url{http://www.ctan.org/tex-archive/info/symbols/comprehensive/}
\begin{table}
\caption{Useful \LaTeX\ symbols}\label{tbl:LaTeX:Symbols}
\centering\begin{tabular}{ccp{0.5cm}cc}\toprule
\LaTeX & Result && \LaTeX & Result \\
\midrule
    \verb+\texttrademark+ & \texttrademark && \verb+\&+ & \& \\
    \verb+\textcopyright+ & \textcopyright && \verb+\{ \}+ & \{ \} \\
    \verb+\textregistered+ & \textregistered && \verb+\%+ & \% \\
    \verb+\textsection+ & \textsection && \verb+\verb!~!+ & \verb!~! \\
    \verb+\textdagger+ & \textdagger && \verb+\$+ & \$ \\
    \verb+\textdaggerdbl+ & \textdaggerdbl && \verb+\^{}+ & \^{} \\
    \verb+\textless+ & \textless && \verb+\_+ & \_ \\
    \verb+\textgreater+ & \textgreater && \\
\bottomrule
\end{tabular}
\end{table}

%%%%%%%%%%%%%%%%%%%%%%%%%%%%%%%%%%%%%%%%%%%%%%%%%%%%%%%%%%%%%%%%%%%%%%
\section{Changing Page Widths and Heights}

The \class{ubcdiss} class is based on the standard \LaTeX\ \class{book}
class~\cite{lamport-1994-ladps} that selects a line-width to carry
approximately 66~characters per line.  This character density is
claimed to have a pleasing appearance and also supports more rapid
reading~\cite{bringhurst-2002-teots}.  I would recommend that you
not change the line-widths!

\subsection{The \texttt{geometry} Package}

Some students are unfortunately saddled with misguided supervisors
or committee members whom believe that documents should have the
narrowest margins possible.  The \latexpackage{geometry} package is
helpful in such cases.  Using this package is as simple as:
\begin{lstlisting}
    \usepackage[margin=1.25in,top=1.25in,bottom=1.25in]{geometry}
\end{lstlisting}
You should check the package's documentation for more complex uses.

\subsection{Changing Page Layout Values By Hand}

There are some miserable students with requirements for page layouts
that vary throughout the document.  Unfortunately the
\latexpackage{geometry} can only be specified once, in the document's
preamble.  Such miserable students must set \LaTeX's layout parameters
by hand:
\begin{lstlisting}
    \setlength{\topmargin}{-.75in}
    \setlength{\headsep}{0.25in}
    \setlength{\headheight}{15pt}
    \setlength{\textheight}{9in}
    \setlength{\footskip}{0.25in}
    \setlength{\footheight}{15pt}

    % The *sidemargin values are relative to 1in; so the following
    % results in a 0.75 inch margin
    \setlength{\oddsidemargin}{-0.25in}
    \setlength{\evensidemargin}{-0.25in}
    \setlength{\textwidth}{7in}       % 1.1in margins (8.5-2*0.75)
\end{lstlisting}
These settings necessarily require assuming a particular page height
and width; in the above, the setting for \verb+\textwidth+ assumes
a \textsc{US} Letter with an 8.5'' width.
The \latexpackage{geometry} package simply uses the page height and
other specified values to derive the other layout values.
The
\href{http://tug.ctan.org/tex-archive/macros/latex/required/tools/layout.pdf}{\texttt{layout}}
package provides a
handy \verb+\layout+ command to show the current page layout
parameters. 


\subsection{Making Temporary Changes to Page Layout}

There are occasions where it becomes necessary to make temporary
changes to the page width, such as to accomodate a larger formula. 
The \latexmiscpackage{chngpage} package provides an \env{adjustwidth}
environment that does just this.  For example:
\begin{lstlisting}
    % Expand left and right margins by 0.75in
    \begin{adjustwidth}{-0.75in}{-0.75in}
    % Must adjust the perceived column width for LaTeX to get with it.
    \addtolength{\columnwidth}{1.5in}
    \[ an extra long math formula \]
    \end{adjustwidth}
\end{lstlisting}


%%%%%%%%%%%%%%%%%%%%%%%%%%%%%%%%%%%%%%%%%%%%%%%%%%%%%%%%%%%%%%%%%%%%%%
\section{Keeping Track of Versions with Revision Control}
\label{sec:DissertationRevisionControl}

Software engineers have used \acf{RCS} to track changes to their
software systems for decades.  These systems record the changes to
the source code along with context as to why the change was required.
These systems also support examining and reverting to particular
revisions from their system's past.

An \ac{RCS} can be used to keep track of changes to things other
than source code, such as your dissertation.  For example, it can
be useful to know exactly which revision of your dissertation was
sent to a particular committee member.  Or to recover an accidentally
deleted file, or a badly modified image.  With a revision control
system, you can tag or annotate the revision of your dissertation
that was sent to your committee, or when you incorporated changes
from your supervisor.

Unfortunately current revision control packages are not yet targetted
to non-developers.  But the Subversion project's
\webref{http://tortoisesvn.net/docs/release/TortoiseSVN_en/}{TortoiseSVN}
has greatly simplified using the Subversion revision control system
for Windows users.  You should consult your local geek.

A simpler alternative strategy is to create a GoogleMail account
and periodically mail yourself zipped copies of your dissertation.

%%%%%%%%%%%%%%%%%%%%%%%%%%%%%%%%%%%%%%%%%%%%%%%%%%%%%%%%%%%%%%%%%%%%%%
\section{Recommended Packages}

The real strength to \LaTeX\ is found in the myriad of free add-on
packages available for handling special formatting requirements.
In this section we list some helpful packages.

\subsection{Typesetting}

\begin{description}
\item[\latexpackage{enumitem}:]
    Supports pausing and resuming enumerate environments.

\item[\latexpackage{ulem}:]
    Provides two new commands for striking out and crossing out text
    (\verb+\sout{text}+ and \verb+\xout{text}+ respectively)
    The package should likely
    be used as follows:
    \begin{verbatim}
    \usepackage[normalem,normalbf]{ulem}
    \end{verbatim}
    to prevent the package from redefining the emphasis and bold fonts.

\item[\latexpackage{chngpage}:]
    Support changing the page widths on demand.

\item[\latexpackage{mhchem}:] 
    Support for typesetting chemical formulae and reaction equations.

\end{description}

Although not a package, the
\webref{http://www.ctan.org/tex-archive/support/latexdiff/}{\texttt{latexdiff}}
command is very useful for creating changebar'd versions of your
dissertation.


\subsection{Figures, Tables, and Document Extracts}

\begin{description}
\item[\latexpackage{pdfpages}:]
    Insert pages from other PDF files.  Allows referencing the extracted
    pages in the list of figures, adding labels to reference the page
    from elsewhere, and add borders to the pages.

\item[\latexpackage{subfig}:]
    Provides for including subfigures within a figure, and includes
    being able to separately reference the subfigures.  This is a
    replacement for the older \texttt{subfigure} environment.

\item[\latexpackage{rotating}:]
    Provides two environments, sidewaystable and sidewaysfigure,
    for typesetting tables and figures in landscape mode.  

\item[\latexpackage{longtable}:]
    Support for long tables that span multiple pages.

\item[\latexpackage{tabularx}:]
    Provides an enhanced tabular environment with auto-sizing columns.

\item[\latexpackage{ragged2e}:]
    Provides several new commands for setting ragged text (\eg forms
    of centered or flushed text) that can be used in tabular
    environments and that support hyphenation.

\end{description}


\subsection{Bibliography Related Packages}

\begin{description}
\item[\latexpackage{bibunits}:]
    Support having per-chapter bibliographies.

\item[\latexpackage{footbib}:]
    Cause cited works to be rendered using footnotes.

\item[\latexpackage{bibentry}:] 
    Support placing the details of a cited work in-line.

\item[\latexpackage{custom-bib}:]
    Generate a custom style for your bibliography.

\end{description}


%%%%%%%%%%%%%%%%%%%%%%%%%%%%%%%%%%%%%%%%%%%%%%%%%%%%%%%%%%%%%%%%%%%%%%
\section{Moving On}
\label{sec:Conclusions}

At this point, you should be ready to go.  Other handy web resources:
\begin{itemize}
\item \webref{http://www.ctan.org}{\ac{CTAN}} is \emph{the} comprehensive
    archive site for all things related to \TeX\ and \LaTeX. 
    Should you have some particular requirement, somebody else is
    almost certainly to have had the same requirement before you,
    and the solution will be found on \ac{CTAN}.  The links to
    various packages in this document are all to \ac{CTAN}.

\item An online
    \webref{http://www.ctan.org/get/info/latex2e-help-texinfo/latex2e.html}{%
	reference to \LaTeX\ commands} provides a handy quick-reference
    to the standard \LaTeX\ commands.

\item The list of 
    \webref{http://www.tex.ac.uk/cgi-bin/texfaq2html?label=interruptlist}{%
	Frequently Asked Questions about \TeX\ and \LaTeX}
    can save you a huge amount of time in finding solutions to
    common problems.

\item The \webref{http://www.tug.org/tetex/tetex-texmfdist/doc/}{te\TeX\
    documentation guide} features a very handy list of the most useful
    packages for \LaTeX\ as found in \ac{CTAN}.

\item The
\webref{http://www.ctan.org/tex-archive/macros/latex/required/graphics/grfguide.pdf}{\texttt{color}}
    package, part of the graphics bundle, provides handy commands
    for changing text and background colours.  Simply changing
    text to various levels of grey can have a very 
    \textcolor{greytext}{dramatic effect}.


\item If you're really keen, you might want to join the
    \webref{http://www.tug.org}{\TeX\ Users Group}.

\end{itemize}

\endinput

Any text after an \endinput is ignored.
You could put scraps here or things in progress.
